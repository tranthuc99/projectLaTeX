\pagenumbering{roman}
\tableofcontents
\pagestyle{fancy}
\chapter{Mở đầu}
\pagenumbering{arabic}
 Cơ cấu kinh tế luôn vận động và biến đổi không ngừng và chịu sự ảnh hưởng của các nhân tố khác đặc biệt là kết cấu hạ tầng kinh tế-xã hội.Thông thường các quốc gia trên thế giới chia các ngành sản xuất, kinh doanh làm 3 nhóm: nông nghiệp, công nghiệp và dịch vụ, khi bị kết cấu hạ tầng kinh tế- xã hội tác động lên sẽ có những thay đổi rõ rệt trong cơ cấu ngành kinh tế. Vì vậy việc nghiên cứu  cơ cấu ngành kinh tế, sự chuyển dịch cơ cấu ngành kinh tế khi bị ảnh hưởng của kết cấu hạ tầng kinh tế - xã hội ở nước ta nhằm tìm ra cách thức phù hợp để phát huy những mặt tốt, thúc đẩy tăng trưởng kinh tế  trong điều kiện nhất định sẵn có ở nước ta.

\chapter{Nội dung}
\section[Lý luận chung]{Một số lý luận chung về kết cấu hạ tầng kinh tế - xã hội, cơ cấu ngành kinh tế và chuyển dịch cơ cấu ngành kinh tế}
\subsection{Kết cấu hạ tầng kinh tế - xã hội}
Kết cấu hạ tầng là một bộ phận đặc thù của cơ sở vật chất kỹ thuật trong nền kinh tế quốc dân có chức năng, nhiệm vụ cơ bản là đảm bảo những điều kiện chung cần thiết cho quá trình sản xuất và tái sản xuất mở rộng được diễn ra bình thường, liên tục.

Nền kinh tế quốc dân là một hệ thống phức tạp được cấu thành từ nhiều bộ phận, do đó, có nhiều cách khác nhau trong việc xem xét kế cấu hạ tầng. Trong mỗi lĩnh vực, mỗi ngành, mỗi khu vực khác nhau thì kết cấu hạ tầng cũng phân chia thành nhiều loại khác nhau:

\begin{itemize}
\item Căn cứ theo sự phân ngành của nền kinh tế quốc dân, thì kết cấu hạ tầng có thể được phân chia thành: kết cấu hạ tầng trong công nghiệp,trong nông nghiệp, trong giao thông vận tải, xây dựng, y tế, giáo dục…
\item Căn cứ theo khu vực dân cư, vùng lãnh thổ thì kết cấu hạ tầng phân chia thành: kết cấu hạ tầng đô thị, kết cấu hạ tầng nông thôn, kết cấu hạ tầng kinh tế biển…
\item Căn cứ theo khu vực dân cư, vùng lãnh thổ thì kết cấu hạ tầng được phân chia thành: kết cấu hạ tầng phục vụ kinh tế, kết cấu hạ tầng phục vụ xã hội và kết cấu hạ tầng pục vụ an ninh – quốc phòng.
\end{itemize}

Nhưng khi nghiên cứu về kết cấu hạ tầng, kết cấu hạ tầng được phân chia thành hai loại cơ bản chính là kết cấu hạ tầng kinh tế và kết cấu hạ tầng xã hội. Kết cấu hạ tầng kinh tế là một bộ phận quan trọng trong hệ thống kinh tế, đảm bảo cho nền kinh tế phát triển nhanh, ổn định, bền vững và là động lực thúc đẩy phát triển nhanh hơn , tạo điều kiện cải thiện cuộc sống dân cư. Kết cấu hạ tầng thuộc loại này bao gồm các công trình kỹ thuật như năng lượng (điện, than, dầu khí) phục vụ sản xuất và đời sống, các công trình giao thông vận tải (đường bộ, đường sắt, đường thủy, đường hàng không, đường ống), bưu chính viễn thông, các công trình thủy lợi phục vụ sản xuất nông – lâm – ngư nghiệp... Kết cấu hạ tầng xã hội là tổng hợp các công trình phục vụ cho các điểm dân cư như nhà văn hóa, các cơ sở y tế, các trường học và các hoạt động dịch vụ công cộng khác. Các công trình này thường gắn liền với đời sống của các điểm dân cư, góp phần ổn định nâng cao đời sống dân cư trên lãnh thổ.

Phát triển kết cấu hạ tầng luôn được đặt ở vị trí trọng tâm trong các chủ trương, đường lối phát triển của Đảng và Nhà nước và luôn là lĩnh vực ưu tiên đầu tư sử dụng Ngân sách Nhà nước. Kết cấu hạ tầng kinh tế - xã hội là nền tảng vật chất có vai trò đặc biệt quan trọng đối với nước ta, nó có tác động mạnh mẽ đến chuyển dịch cơ cấu ngành kinh tế ở nước ta.

\subsection{Cơ cấu ngành kinh tế}
Cơ cấu ngành kinh tế là tập hợp các bộ phận hợp thành của tổng thể nền kinh tế và mối tương quan tỷ lệ giữa các bộ phận hợp thành so với tổng thể. Cơ cấu ngành kinh tế phản ánh phần nào trình độ phân công lao động xã hội chung của nền kinh tế và trình độ phát triển chung của lực lượng sản xuất.

Có thể xem xét cơ cấu của nền kinh tế trên các phương diện như cơ cấu ngành kinh tế, cơ cấu vùng kinh tế và cơ cấu thành phần kinh tế. Cơ cấu ngành kinh tế là cơ cấu kinh tế trong đó mỗi bộ phận hợp thành là một ngành hay một nhóm ngành kinh tế.

Nói cách khác, cơ cấu ngành kinh tế là tập hợp các ngành ( các nhóm ngành ) hợp thành nên tổng thể nền kinh tế và mối quan hệ của mỗi ngành trong tổng thể.

Cơ cấu ngành kinh tế thường được xem xét theo 3 nhóm ngành chính: Nông nghiệp, công nghiệp và dịch vụ và trong nội bộ mỗi nhóm ngành lại có cơ cấu riêng. Nó tập hợp tất cả các ngành hình thành nên nền kinh tế và các mối quan hệ tương đối ổn định giữa chúng. Trong cơ cấu ngành kinh tế của các nước phát triển thì dịch vụ, công nghiệp chiếm tỉ lệ cao, còn đối với các nước đang phát triển nông nghiệp chiếm tỉ lệ cao mặc dù công nghiệp, dịch vụ đã tăng.

\subsection{Chuyển dịch cơ cấu ngành kinh tế}
Chuyển dịch cơ cấu ngành kinh tế là sự vận động phát triển của các ngành làm thay đổi vị trí, tỷ trọng và mối quan hệ tương tác giữa chúng theo thời gian để phù hợp với sự phát triển ngày càng cao của lực lượng sản xuất và phân công lao động xã hội. Chuyển dịch cơ cấu ngành kinh tế luôn là vấn đề then chốt, đóng vai trò quyết định đối với quá trình tăng trưởng và phát triển kinh tế.

Cơ cấu ngành kinh tế luôn thay đổi theo từng thời kì phát triển bởi các ngành kinh tế luôn có sự chuyển biến về quy mô và tỷ trọng do tốc độ tăng trưởng của các ngành là khác nhau. Sự biến đổi của cơ cấu ngành kinh tế là để phù hợp với trình độ phát triển của lực lượng sản xuất và phân công lao động xã hội. 

Chuyển dịch cơ cấu ngành kinh tế nhằm: phát huy các lợi thế so sánh để khai thác và sử dụng có hiệu quả nguồn lực phát triển của quốc gia; tạo ra khả năng sản xuất hàng hóa với khối lượng lớn hơn, chất lượng cao hơn, đa dạng hóa về chủng loại đáp ứng nhu cầu trong nước và xuát khẩu; góp phần tạo ra nhiều việc làm và tăng thu nhập, nâng cao mức sống cho người lao động; góp phần nâng cao năng lực cạnh tranh của nền kinh tế.

\section[Thực trạng, phân tích, đánh giá]{Thực trạng, phân tích, đánh giá ảnh hưởng của kết cấu hạ tầng kinh tế - xã hội đến chuyển dịch cơ cấu ngành kinh tế ở Việt Nam}

\subsection{Thực trạng}
Kết cấu hạ tầng kinh tế - xã hội tác động lớn đến chuyển dịch cơ cấu ngành kinh tế, là nhân tố hỗ trợ thúc đẩy phát triển nền kinh tế theo hướng công nghiệp hóa, hiện đại hóa. Cơ cấu ngành kinh tế của nước ta đang dần thay đổi khác nhau, các ngành kinh tế có sự chuyển biến về quy mô và tỷ trọng qua các năm. Đảng và Nhà nước đã có những chủ trương, chính sách để tác động lên cơ cấu ngành kinh tế trong điều kiện tận dụng những lợi thế và nguồn lực sẵn có. Trong đó, không quên chú trọng đầu tư, phát triển kết cấu hạ tầng kinh tế - xã hội cũng ảnh hưởng không nhỏ đến chuyển dịch cơ cấu ngành kinh tế. Có thể thấy rằng trong nhiều năm qua, Việt Nam đang cố gắng vươn mình trở thành một nước công nghiệp hóa, hiện đại hóa. Vì thế nước ta đã và đang thực hiện những kế hoạch, mục tiêu đã đề ra, tăng cường đầu tư, phát triển cho những nền tảng vật chất để phục vụ quá trình đi lên trở thành một nước công nghiệp mới.

Sau 30 năm đổi mới, nước ta đã đạt được những kết quả nổi bật: tốc độ tăng liên tục và khá ổn định của GDP, cơ cấu ngành kinh tế đã có sự thay đổi đáng kể theo hướng tích cực. Trong các ngành kinh tế và dịch vụ có đóng góp lớn cho tăng GDP. Điều này hoàn toàn phù hợp với xu hướng chuyển dịch cơ cấu kinh tế trong quá trình công nghiệp hóa, hiện đại hóa đất nước.Chuyển dịch cơ cấu ngành kinh tế đã làm thay đổi cơ cấu lao động theo ngành ở nước ta theo hướng công nghiệp hóa, hiện đại hóa. Tỷ trọng lao động trong các ngành công nghiệp và dịch vụ ngày càng tăng lên, trong khi tỷ trọng lao động ngành nông nghiệp ngày càng giảm đi.

Hệ thống kết hạ tầng kinh tế - xã hội phát triển đồng bộ, hiện đại sẽ thúc đẩy tăng trưởng kinh tế, nâng cao năng suất, hiệu quả của nền kinh tế và đóng góp phần giải quyết các vấn đề xã hội. Vì vậy, nước ta đã đầu tư những khoản rất lớn cho xã hội như các công trình công cộng, trường học, các bệnh viện, cơ sở y tế và các điểm dân cư hay những cơ sở vật chất kinh tế để phục vụ phát triển nền kinh tế như các công trình giao thông vận tải, các nhà máy nhiệt điện, thủy điện, các công trình thủy lợi, bưu chính viễn thông…

Tuy nhiên, hệ thống kết cấu hạ tầng kinh tế - xã hội kém phát triển gây ra một trở lực lớn đến chuyển đổi cơ cấu ngành kinh tế, có thể thấy rõ điều này ở các nước đang phát triển nói chung và nước ta nói riêng. Mặc dù nước ta đang tích cực đầu tư phát triển cho kết cấu hạ tầng kinh tế - xã hội nhưng vẫn còn nhiều hạn chế dẫn đến những bất cập, gây ảnh hưởng không nhỏ trong quá trình phát triển kinh tế và chuyển dịch cơ cấu ngành kinh tế. Cụ thể là các công trình với quy mô hàng nghìn tỷ đồng xây xong bỏ hoang hoặc ít phát huy hiệu quả trong đời sống như tòa ký túc xá 9 tầng cùng các tòa khác ở cơ sở 2 của đại học Sao Đỏ nhiều năm không sử dụng đã bị phong rêu phủ kín…

Nội dung và yêu cầu cơ bản của chuyển dịch cơ câu kinh tế ở nước ta theo hướng công nghiệp hóa, hiện đại hóa. Kết cấu hạ tầng kinh tế - xã hội ảnh hưởng tới sự hình thành và phát triển kinh tế, chuyển dịch cơ cấu ngành kinh tế nước ta theo hướng tăng nhanh tỷ trọng giá trị GDP của nhóm ngành công nghiệp và dịch vụ, đồng thời giảm tương đối tỷ trọng giá trị GDP của nhóm ngành nông nghiệp. Cùng với quá trình chuyển dịch cơ cấu  kinh tế tất yếu sẽ dẫn đến những biến đổi kinh tế và xã hội theo hướng công nghiệp hóa, hiện đại hóa của cơ cấu các vùng kinh tế, các thành phần kinh tế, các lực lượng lao động xã hội, cơ cấu kinh tế đối nội, cơ cấu kinh tế đối ngoại…

\subsection{Phân tích và đánh giá}
Phát triển kết cấu hạ tầng là điều kiện tiên quyết để phát triển kinh tế, kết cấu hạ tầng kinh tế - xã hội phát triển sẽ đảm bảo cho kinh tế hàng hóa phát triển, nâng cao đời sống vật chất và tinh thần cho dân cư. Kết cấu hạ tầng kinh tế - xã hội ảnh hưởng lớn tới sự hình thành và phát triển của các ngành kinh tế, chi phối trình độ kỹ thuật và công nghệ,… do đó, nó là một nhân tố quan trọng tác động đén sự hình thành, vận động và biến đổi của cơ cấu kinh tế.

Trước đây nước ta là một nước nông nghiệp lạc hậu, có kết cấu hạ tầng đơn sơ, kém phát triển. Đặc biệt trong thời kì Pháp thuộc, thực dân Pháp ra sức kìm hãm, hạn chế cho xây dựng, đầu tư vào kết cấu hạ tầng kinh tế xã hội để nước ta không thể phát triển hơn chính quốc. Hay hiện nay giữa đô thị và nông thôn ở nước ta có thể thấy sự khác biệt rõ rệt trong cơ cấu ngành kinh tế, lý do bởi kết cấu hạ tầng ở đô thị phát triển hơn so với nông thôn. Chính điều này đã cản trở sự giao thương phát triển trong sản xuất của nước ta. Ngày nay Đảng và Nhà nước đã chỉ ra tầm quan trọng trong việc đầu tư vào kết cấu hạ tầng kinh tế- xã hội:

-Kết cấu hạ tầng kinh tế được đầu tư hàng nghìn tỷ đồng, đặc biệt hệ thống đường giao thông rất được quan tâm và chú trọng. Giao thông vận tải có tốt thì giao thương mới thuận lợi, phải mở rộng thị trường mới. Từ đây chúng ta có thể tăng cường xuất nhập khẩu, thu hút vốn đầu tư từ nước ngoài, mở rộng thị trường, tiếp thu những đổi mới trong khoa học công nghệ. Vì vậy đây chính là cánh cửa mở rộng giúp chúng ta phát triển công nghiệp, dịch vụ.

Ngoài ra, việc xây dựng các nhà máy điện, công trình thủy lợi rất quan trọng vì nó cung cấp năng lượng phục vụ cho sản xuất kinh doanh, nâng cao năng xuất sản xuất trong nhiều lĩnh vực. Các công trình thủy lợi, máy móc cơ khí hiện đại làm tăng năng xuất thu hoạch trong nông nghiệp mặc dù tỉ trọng ngành này đã giảm đi nhiều trong cơ cấu GDP. Máy móc thiết bị hiện đại đã thay thế một phần lao động trong nông nghiệp giúp họ chuyển sang sản xuất trong nhóm ngành công nghiệp, dịch vụ. Tăng nhanh năng lực và hiện đại hóa bưu chính viễn thông, tiếp tục phát triển mạng thông tin liên lạc quốc gia, mở liên lạc điện thoại đến hầu hết các xã, kết nối internet toàn cầu, giúp kết nối với mọi nơi toàn thế giới một cách dễ dàng. Với các gói dịch vụ tiện ích đơn giản được mọi người đón nhận và ngày càng trở nên phổ biến rộng rãi góp phần nào đó phát triển ngành dịch.

Các nhân tốt trong kế cấu hạ tầng kinh tế có mối quan hệ khăng khít, có tính liên kết cao, tác động qua lại lẫn nhau. Kết cấu hạ tâng kinh tế được xem xét trong mối quan hệ giữa từng ngành với hệ thống kết cấu hạ tầng kinh tế nói chung: chẳng hạn ngành công nghiệp, nông nghiệp, hay dịch vụ đều chỉ có thể vận hành thực sự hiệu quả khi mà hệ thống giao thông vận tải, thông tin liên lạc, hệ thống cung cấp điện nước phát triển

Chuyển dịch cơ cấu kinh tế theo hướng công nghiệp hóa hiện đại hóa là tăng tỉ trọng giá trị trong GDP của nhóm ngành công nghiệp, dịch vụ. Để tăng giá trị của các nhóm ngành này, cần phải đầu tư phát triển cơ sở hạ tầng hiện đại, có máy móc sản xuất tiên tiến để tạo ra những sản phẩm công nghiệp dịch vụ tốt, kích thích tiêu dùng. Sản phẩm tốt được người tiêu dùng đón nhận làm tác động thúc đẩy tiếp tục sản xuất, mở rộng quy mô, tỷ trọng trong cơ cấu ngành kinh tế.

- Kết cấu hạ tầng xã hội cũng ảnh hưởng không nhỏ đến chuyển dịch cơ cấu ngành kỉnh tế của nước ta. Con người là nhân tố quan trọng trong quyết định mọi hoạt động sản xuất kinh doanh. Vì vậy, xây dựng hệ thống các công trình giáo dục, y tế thể dục, thể thao, nhà ở của dân cư là vô cùng cần thiết nhằm nâng cao đời sống cộng đồng. Con người có thể thụ hưởng các dịch vụ chăm sóc sức khỏe, học tập, rèn luyện và điều kiện làm việc tốt nhất. Nhờ những nhu cầu về vật chất và tinh thần này mà ngành dịch vụ ngày càng được quan tâm và phát triển. Do đó tỉ trọng ngành dịch vụ cũng ngày càng tăng. 

Kết cấu hạ tầng kinh tế-xã hội càng được quan tâm và chú trọng thì kinh tế càng phát triển, giúp chuyển dịch cơ cấu kinh tế nước ta theo hướng tích cực. Nhìn chung, kết cấu hạ tầng là hệ thống các công trình phức tạp với yêu cầu vốn đầu tư lớn, là những công trình phục vụ đa mục đích, nhất là chuyển dịch cơ cấu ngành kinh tế ở nước ta theo hướng công nghiệp hóa, hiện đại hóa để nước ta nhanh chóng thoát khỏi tình trạng lạc hậu, chậm phát triển, trở thành một quốc gia văn minh, hiện đại.

Khi kết cấu hạ tầng kinh tế-xã hội không phát triển thì sẽ làm giảm sức hấp dẫn của môi trường kinh doanh, cản trở việc gia tăng đầu tư dẫn đến cơ cấu kinh tế của nước ta không có những chuyển biến tích cực

Ví dụ rõ nhất ở các nước phát triển như Đức, Mỹ, Nhật Bản, từ những năm 50, 60 của thế kỉ XX, họ đã rất chú trọng đầu tư cho kết cấu hạ tầng kinh tế- xã hội và thấy rằng việc phát triển kết cấu hạ tầng kinh tế - xã hội đã kéo theo mức giảm của chi phí sản xuất trong các ngành công nghiệp, đặc biệt là ngành công nghiệp chế biến. Và chính chi phí sản xuất, chi phí vận chuyển giảm họ đã đạt được những thành công trong ngành công nghiệp và sẵn sàng chi một khoản tiền lớn để đầu tư cho hệ thống giao thông vận tài, nhà máy, máy móc hiện đại…

Bên cạnh đó việc đầu tư vào kết cấu hạ tầng mà không có kế hoạch hợp lý sẽ không đạt được hiệu quả mong muốn và gây lãng phí tiền bạc cho Nhà nước và xã hội. Còn tồn tại tình trạng đầu tư dàn trải, nhiều công trình dở dang. Đầu tư chưa đồng bộ giữa các phân ngành kết cấu hạ tầng và trong nội bộ từng ngành. Trong cơ cấu đầu tư, chưa dành tỷ lệ vốn thích đáng cho công tác bảo trì, bảo dưỡng hệ thống mạng, không bảo đảm phát triển bền vững, tình trạng xuống cấp tiếp tục diễn ra. Khi các kết cấu hạ tầng kinh tế - xã hội hoạt động không hiệu quả gây ảnh hưởng xấu đến quá trình sản xuất kinh doanh. Do đó, các nhóm ngành kinh tế sẽ gặp nhiều khó khăn trong khâu sản xuất và đưa sản phẩm đến tay người tiêu dùng. Muốn chuyển dịch cơ cấu ngành kinh tế nước ta theo hướng công nghiệp hóa, hiện đại hóa phải khắc phục những bất cập, hạn chế để tạo điều kiện thuận lợi phát triển kinh tế.

Phát triển kết cấu hạ tầng tạo được vật chất- kĩ thuật cho đổi mới công nghệ theo hưởng hiện đại nhằm phát triển mạnh mẽ lực lượng sản xuất và xây dựng cơ cấu kinh tế hợp lý, nâng cao đời sống vật chất, tinh thần cho người dân. Phát triển kế cấu hạ tầng thúc đẩy phân công lại lao động xã hội, chuyển dịch cơ cấu ngành kinh tế nước ta, giúp mở rộng thị trường và nâng coa khả nẳng cạnh tranh của ngành( do giảm chi phí và tăng lợi nhuận). Do đó, làm chuyển dịch cơ cấu kinh tế theo hướng đáp ứng nhu cầu của thị trường.

Như vậy, kết cấu hạ tầng vừa là điều kiện, tiên đề của công nghiệp hóa, hiện đại hóa, vừa là kết quả trực tiếp của công nghiệp hóa, hiện đại hóa đất nước.

\section[Giải pháp phát triển  kết cấu hạ hạ tầng kinh tế - xã hội]{Giải pháp phát triển  kết cấu hạ hạ tầng kinh tế - xã hội nhằm chuyển dịch cơ cấu ngành kinh tế nước ta theo hướng công nghiệp hóa, hiện đại hóa}

\subsection{Phát triển hình thành hệ thống giao thông rộng khắp lãnh thổ cả nước, nối các vùng khó khăn với các vùng kinh tế trọng điểm và trung tâm đô thị lớn, phát triển hệ thống giao lưu quốc tế}

- Nước ta có mạng lưới sông ngòi dày đặc, có bờ biển dài, vùng biển rộng lớn, thuận lợi phát triển nhanh các ngành kinh tế hàng hải. Vì vậy chúng ta nên đầu tư để xây dựng các cảng biển, mở thêm các tuyến đường thủy giúp thuận lợi cho việc giao thương, chuyên chở, bốc dỡ hàng hóa giữa các vùng và các nước với nhau.

- Về đường bộ, cần tu sửa, ngăn chặn sự xuống cấp, từng bước nâng cấp những tuyến đường trọng yếu, hình thành mạng lưới giao thông đồng bộ giữa các vùng trọng điểm. Thực hiện  xây dựng các dự án giao thông nông thôn phù hợp với đặc điểm từng vùng.

- Phát triển đường hàng không để giao lưu quốc tế, nâng cấp hệ thống đường sắt của quốc gia.

\subsection{Phát triển các công trình thủy lợi ở nước ta}
\begin{itemize}
\item[-] Nâng cao hiệu quả khai thác các hệ thống thủy lợi hiện có như hồ chứa, đê điều, hệ thống kênh rạch, cầu cống… phục vụ đa mục tiêu là giải pháp nhanh và kinh tế nhất, phục vụ chuyển đổi cơ cấu ngành nông nghiệp hiệu quả. 

\item[-] Tiếp tục đầu tư xây dựng mới các hệ thống công trình thủy lợi như hệ thống đê phòng chống thiên tai, đặc biệt là hệ thống tưới cho cây trồng cạn, hệ thống thủy lợi phục vụ nuôi trồng thủy hải sản, làm muối. Đây là những lĩnh vực thế mạnh của nước ta nên cần được chú trọng đầu tư. Vấn đề xử lý nước thải từ các khu nuôi trồng và vấn đề tiêu thoát nước thải đảm bảo môi trường bền vững nhất thiết phải quan tâm đày đủ.

\item[-] Tiếp tục bổ sung các thể chế từ quy phạm, quy trình quản lý hệ thống đến các công trình cụ thể tạo điều kiện cho cán bộ công nhân trực tiếp vận hành công trình và hệ thống một cách khoa học và thuận lợi.

\item[-] Tu sửa, nâng cấp các công trình thủy lợi đã xuống cấp để hoạt động một cách hiệu quả nhất.
\end{itemize}


\subsection{Phát triển hạ tầng bưu chính – viễn thông}
\begin{itemize}
\item[-] Tăng cường công tác tuyên truyền nâng cao nhận thức cho người dân về sim thuê bao di động trả trước, số hóa truyền hình, vai trò hạ tầng viễn thông đối với phát triển kinh tế, văn hóa, xã hội.

\item[-] Rà soát và thực hiện kiểm định chất lượng nhằm đảm bảo an toàn công trình viễn thông theo quy định hiện hành, tạo môi trường cạnh tranh lành mạnh trong kinh doanh phát triển dịch vụ viễn thông, thực hiện chỉnh trang cáp viễn thông tại các khu vực trung tâm, khu đô thị và thực hiện treo biển hiệu cáp của doanh nghiệp…

\item[-] Bảo dưỡng các hệ thống thiết bị, đường truyền, xây dựng các dự án tối ưu dự phòng phân tải, ứng cứu tình huống, sự cố bất thường gây tắc nghẽn, mất liên lạc của mạng viễn thông…

\end{itemize}

\subsection{Đầu tư xây dựng các cơ sở y tế}
\begin{itemize}
\item[-] Tổ chức vận động người dân tham gia bảo hiểm y tế, đặc biệt cần giúp đỡ người dân những vùng khó khăn được hưởng các dịch vụ chăm sóc sức khỏe.

\item[-] Đầu tư nâng cao trình độ, năng lực cho bác sĩ thực hành tổng quát và điều dưỡng đồng thời nâng cấp cơ sở vật chất y tế tạo điều kiện thăm khám chữa bệnh tốt nhất cho người dân.

\item[-] Đầu tư cho nghiên cứu các loại vắc–xin phòng chống dịch bệnh, không ngừng cố gắng để phát triển, nâng cao y tế nước nhà.
\end{itemize}


\subsection{Phát triển giáo dục}
\begin{itemize}
\item[-] Tiếp tục hoàn hiện, đẩy mạnh cải cách hành chính về GD-ĐT, Các địa phương cần rà soát , quy hoạch mạng lưới trường lớp phù hợp, tạo điều kiện cho người dân tham gia học tập.

\item[-] Nâng cao chất lượng đội ngũ cán bộ giáo viên và cán bộ quản lý giáo dục các cấp, sắp xếp đội ngũ cán bộ giáo viên hợp lý để tránh xảy ra tình trạng thừa, thiếu giáo viên cục bộ.

\item[-] Nâng cấp cơ sở vật chất trường học, nâng cao chất lượng giáo dục, triển khai chương trình, sách giáo khoa giáo dục phổ thông mới.


\end{itemize}
Tóm lại: Đây là những yếu tố tiêu biểu và quan trọng cần chú trọng và đầu tư. Khi những nền tảng vật chất này phát triển mới tạo điều kiện thuận lợi cho phát triển cơ cấu ngành kinh tế. Kết cấu hạ tầng kinh tế-xã hội có ối quan hệ khăng khít, tác động và ảnh hưởng đến chuyển dịch cơ cấu ngành kinh tế ở nước ta theo hướng công nghiệp hóa, hiện đại hóa.



\chapter{Tổng kết}

Quá trình chuyển dịch cơ cấu ngành kinh tế diễn ra như thế nào phụ thuộc vào nhiều yếu tố như quy mô nền kinh tế, dân số của quốc gia, các lợi thế về tự nhiên, nhân lực, điều kiện kinh tế, xã hội. Và trong đó, phát triển kết cấu kinh tế- xã hội đóng vai trò rất quan trọng, là điều kiện tiên quyết để phát triển kinh tế. Kết cấu hạ tầng phát triển sẽ đảm bảo cho kinh tế hàng hóa phát triển, nâng cao đời đới sống vật chất và tinh thần cho dân cư. Như vậy, có thể thấy rằng, kết cấu hạ tầng có vai trò quyết định đối với nền kinh tế của một quốc gia. Một quốc gia phát triển không thể không đầu tư cho việc ổn định và phát triển kết cấu hạ tầng và ngược lại khi có một kết cấu hạ tầng kinh tế - xã hội tốt thì mới có thể chuyển dịch cơ cấu ngành kinh tế theo hướng công nghiệp hóa, hiện đại hóa.

Với quyết tâm cao của Chính phủ trong điều hành đẩy mạnh chuyển dịch cơ cấu kinh tế  một cách hợp lý để đạt mục tiêu đề ra. Dù không có một “con đường”sẵn có, một “công thức” chung nào cho tất cả các nền kinh tế trong quá trình chuyển dịch cơ cấu ngành kinh tế nhưng nước ta vẫn không ngừng học hỏi, phát triển, tích lũy kinh nghiệm, khắc phục các khó khăn để đưa ra nhưng phương pháp, kế hoạch phù hợp với nền kinh tế nước nhà. Và ta có thể thấy rõ kết cấu hạ tầng kinh tế - xã hội ở nước ta ngày càng được nâng cao kéo theo những chuyển biến tích cực trong cơ cấu ngành kinh tế ở nước ta là tỷ trọng các ngành công nghiệp, dịch vụ ngày càng tăng nhanh, giảm tỷ trọng nhóm ngành nông nghiệp. Đây là những thành công để nước ta trở thành một nước công nghiệp mới trong tương lai gần

\begin{thebibliography}{9}
\bibitem{latexcompanion} 
Wikipedia - Bách khoa toàn thư mở \\ \texttt{https://www.wikipedia.org/} 

\bibitem{einstein} 
Giáo trình Kinh tế phát triển, NXB Tài Chính

%\bibitem{knuthwebsite} 
%Knuth: Computers and Typesetting,
%\\\texttt{http://www-cs-faculty.stanford.edu/\~{}uno/abcde.html}
\end{thebibliography}