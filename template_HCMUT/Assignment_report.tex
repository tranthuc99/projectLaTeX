\documentclass[a4paper]{article}
\usepackage{vntex}
%\usepackage[english,vietnam]{babel}
%\usepackage[utf8]{vietnam}

%\usepackage[utf8]{inputenc}
%\usepackage[francais]{babel}
\usepackage{a4wide,amssymb,epsfig,latexsym,multicol,array,hhline,fancyhdr}
\usepackage{booktabs}
\usepackage{amsmath}
\usepackage{lastpage}
\usepackage[lined,boxed,commentsnumbered]{algorithm2e}
\usepackage{enumerate}
\usepackage{color}
\usepackage{graphicx}							% Standard graphics package
\usepackage{array}
\usepackage{tabularx, caption}
\usepackage{multirow}
\usepackage[framemethod=tikz]{mdframed}% For highlighting paragraph backgrounds
\usepackage{multicol}
\usepackage{rotating}
\usepackage{graphics}
\usepackage{geometry}
\usepackage{setspace}
\usepackage{epsfig}
\usepackage{tikz}
\usepackage{listings}
\usetikzlibrary{arrows,snakes,backgrounds}
\usepackage{hyperref}
\hypersetup{urlcolor=blue,linkcolor=black,citecolor=black,colorlinks=true} 
%\usepackage{pstcol} 								% PSTricks with the standard color package

\newtheorem{theorem}{{\bf Định lý}}
\newtheorem{property}{{\bf Tính chất}}
\newtheorem{proposition}{{\bf Mệnh đề}}
\newtheorem{corollary}[proposition]{{\bf Hệ quả}}
\newtheorem{lemma}[proposition]{{\bf Bổ đề}}

\everymath{\color{blue}}
%\usepackage{fancyhdr}
\setlength{\headheight}{40pt}
\pagestyle{fancy}
\fancyhead{} % clear all header fields
\fancyhead[L]{
 \begin{tabular}{rl}
    \begin{picture}(25,15)(0,0)
    \put(0,-8){\includegraphics[width=8mm, height=8mm]{hcmut.png}}
    %\put(0,-8){\epsfig{width=10mm,figure=hcmut.eps}}
   \end{picture}&
	%\includegraphics[width=8mm, height=8mm]{hcmut.png} & %
	\begin{tabular}{l}
		\textbf{\bf \ttfamily Trường Đại Học Bách Khoa Tp.Hồ Chí Minh}\\
		\textbf{\bf \ttfamily Khoa Khoa Học và Kỹ Thuật Máy Tính}
	\end{tabular} 	
 \end{tabular}
}
\fancyhead[R]{
	\begin{tabular}{l}
		\tiny \bf \\
		\tiny \bf 
	\end{tabular}  }
\fancyfoot{} % clear all footer fields
\fancyfoot[L]{\scriptsize \ttfamily Bài tập lớn môn Mô Hình Hoá Toán Học - Niên khóa 2016-2017}
\fancyfoot[R]{\scriptsize \ttfamily Trang {\thepage}/\pageref{LastPage}}
\renewcommand{\headrulewidth}{0.3pt}
\renewcommand{\footrulewidth}{0.3pt}


%%%
\setcounter{secnumdepth}{4}
\setcounter{tocdepth}{3}
\makeatletter
\newcounter {subsubsubsection}[subsubsection]
%\renewcommand\thesubsubsubsection{\thesubsubsection .\@alph\c@subsubsubsection}
%\newcommand\subsubsubsection{\@startsection{subsubsubsection}{4}{\z@}%
 %                                    {-3.25ex\@plus -1ex \@minus -.2ex}%
  %                                   {1.5ex \@plus .2ex}%
   %                                  {\normalfont\normalsize\bfseries}}
\newcommand*\l@subsubsubsection{\@dottedtocline{3}{10.0em}{4.1em}}
\newcommand*{\subsubsubsectionmark}[1]{}
\makeatother

\definecolor{dkgreen}{rgb}{0,0.6,0}
\definecolor{gray}{rgb}{0.5,0.5,0.5}
\definecolor{mauve}{rgb}{0.58,0,0.82}

\lstset{frame=tb,
	language=Matlab,
	aboveskip=3mm,
	belowskip=3mm,
	showstringspaces=false,
	columns=flexible,
	basicstyle={\small\ttfamily},
	numbers=none,
	numberstyle=\tiny\color{gray},
	keywordstyle=\color{blue},
	commentstyle=\color{dkgreen},
	stringstyle=\color{mauve},
	breaklines=true,
	breakatwhitespace=true,
	tabsize=3,
	numbers=left,
	stepnumber=1,
	numbersep=1pt,    
	firstnumber=1,
	numberfirstline=true
}

\begin{document}

\begin{titlepage}
\begin{center}
ĐẠI HỌC QUỐC GIA THÀNH PHỐ HỒ CHÍ MINH \\
TRƯỜNG ĐẠI HỌC BÁCH KHOA \\
KHOA KHOA HỌC - KỸ THUẬT MÁY TÍNH 
\end{center}

\vspace{1cm}

\begin{figure}[h!]
\begin{center}
\includegraphics[width=3cm]{hcmut.png}
\end{center}
\end{figure}

\vspace{1cm}


\begin{center}
\begin{tabular}{c}
	\multicolumn{1}{l}{\textbf{{\Large MÔ HÌNH HOÁ TOÁN HỌC}}}\\
	~~\\
	\hline
	\\
	\multicolumn{1}{l}{\textbf{{\Large Xử lý tối ưu bài toán}}}\\
	\\
	
	\textbf{{\Huge Hospitals \& Residents trên Matlab}}\\
	\\
	\hline
\end{tabular}
\end{center}

\vspace{3cm}

\begin{table}[h]
\begin{tabular}{rrl}
\hspace{5 cm} & GVHD: &Lê Hồng Trang\\
& SV: & Chìu Tuấn Bình - 1510221\\
& & Mai Đức Tú - 1513924 \\
& & Phồng Quang Tuấn - 1513865\\
& & Lê Duy Hiển - 1511057 \\
& & Nguyễn Đỗ Đức Anh - 1510062\\
\end{tabular}
\end{table}

\begin{center}
{\footnotesize TP. HỒ CHÍ MINH, THÁNG 4/2017}
\end{center}
\end{titlepage}


\thispagestyle{empty}

\newpage
\tableofcontents
\newpage

%%%%%%%%%%%%%%%%%%%%%%%%%%%%%%%%%


%%%%%%%%%%%%%%%%%%%%%%%%%%%%%%%%%

		\begin{mdframed}[hidealllines=true,backgroundcolor=magenta!10]
		\begin{lstlisting}
		% ------------------------------- %
		%     XOA MAN HINH VA CAC BIEN    %
		% ------------------------------- %
		clear
		clc
		
		% ------------------------------- %
		%      NHAP DU LIEU BAI TOAN      %
		% ------------------------------- %
		n = ...;      % So nguoi dan
		m = ...;      % So benh vien
		% Ma tran D bieu dien thu tu uu tien cua benh vien doi voi benh nhan
		% ung voi tung hang
		D = [...];
		% Ma tran B bieu dien thu tu uu tien cua benh nhan doi voi benh vien
		% ung voi tung cot
		B = [...];
		% Ma tran c bieu dien suc chua cua tung benh vien
		c = [...];
		% Ma tran a bieu dien moi benh nhan chi duoc chon lua mot benh vien
		a = ones(n,1);
		
		% ------------------------------- %
		% GIAI BAI TOAN BANG SOLVER MOSEK %
		% ------------------------------- %
		cvx_begin
			cvx_solver mosek
			% Bien x(i,j) chi nhan gia tri 0 hoac 1
			% ung voi su ghep goi benh nhan r_i voi benh vien h_j
			variable x(n,m) binary
			% Toi da tong cac bien x(i,j)
			% tuc la cang nhieu cap duoc ghep doi cang tot
			maximize( 0 )
			subject to
				% Tong cac hang trong cung mot cot (so benh nhan duoc chon)
				% nho hon hoac bang suc chua cua benh vien
				sum(x,1) <= c;
				% Tong cac cot trong cung mot cot (so benh vien duoc chon)
				% nho hon hoac bang 1
				sum(x,2) <= a;
			% Bao dam khong co cac cap chan
			for u = 1:n
				for v = 1:m
					%Tinh so hang dau tien trong ham dieu kien on dinh
					t1 = 0;
					for j = 1:m 
						t1 = t1 + lt(D(u,j),D(u,v)) * x(u,j); 
					end
					%Tinh so hang thu hai trong ham dieu kien on dinh
					t2 = 0;
					for i = 1:n
						t2 = t2 + lt(B(i,v),B(u,v)) * x(i,v) / c(v); 
					end
					%Xac lap ham dieu kien on dinh
					t1 + t2 + x(u,v) >= 1;
					%Ham dam bao cac cap (r_u,h_v) duoc xet nam trong A, neu
					%cap do khong nam trong A thi x_uv = 0
					if D(u,v) == m+n+1 || B(u,v) == m+n+1
						(eq(D(u,v),m+n+1) + eq(B(u,v),m+n+1)) * x(u,v) == 0;
					end
				end
			end
		cvx_end
		
		% ------------------------------- %
		%  HIEN THI KET QUA RA MAN HINH   %
		% ------------------------------- %
		D
		B
		c
		x       % Cac cap duoc ghep doi
	\end{lstlisting}
	\end{mdframed}

	
%%%%%%%%%%%%%%%%%%%%%%%%%%%%%%%%%
\begin{thebibliography}{80}

\bibitem{CVX}
CVX Introduction
``\textbf{link: http://cvxr.com/cvx/doc/intro.html/}'',
\textit{What is CVX}, lần truy cập cuối: 15/04/2017.

\end{thebibliography}
\end{document}

