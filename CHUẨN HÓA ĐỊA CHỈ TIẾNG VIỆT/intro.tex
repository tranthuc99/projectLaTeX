




\vspace*{2cm}
\centerline{\bf \large\MakeUppercase{Lời mở đầu}}
\vspace{20pt}
Bài toán chuẩn hóa địa chỉ, hay còn được biết đến phổ biến trên thế giới với cái tên Address Parser hoặc Address Normalization, được các các công ty lớn, các nhà phát triển trên thế giới quan tâm và nghiên cứu. Điển hình như Google, trong quá trình phát triển Google Maps, họ cũng đã giải quyết bài toán này, tuy nhiên phương pháp thì không được công bố. 

Bài toán chuẩn hóa địa chỉ dựa vào một chuỗi văn bản, có thể chứa địa chỉ, từ đó bóc tách ra được các đơn vị nhỏ hơn cấu thành địa chỉ như tòa nhà, đường, thành phố ... hoặc có thể một số đơn vị khác khi với địa chỉ của nước ngoài.

Bài toán chuẩn hóa địa chỉ có rất nhiều cách tiếp cận, đơn giản nhất có thể sử dụng luật, kết hợp vị trí tiền tố  trong chuỗi văn bản từ đó xác định ra các nhãn từ cần tìm. Gần đây, với sự bùng nổ của học máy, học sâu, bài toán chuẩn hóa địa chỉ có thể được coi là một bài toán gán nhãn chuỗi (sequence labeling) mà qua đó, việc xác định các đơn vị có thể dựa trên một thuật toán học máy, học sâu và một tập dữ liệu huấn luyện.

Vài năm gần đây, các nhà phát triển ở Ấn Độ đã công bố một bài báo về bài toán chuẩn hóa địa chỉ, trong đó họ chỉ ra các nhược điểm của phương pháp sử dụng trí tuệ nhân tạo và đề xuất một phương pháp có tên gọi là SAGEL, một phương pháp theo kết quả đánh giá từ bài báo thì khá ấn tượng với tập địa chỉ của Ấn Độ.

Trong phạm vi của đồ án, tôi đã cố gắng để thử nghiệm một vài phương pháp đã được đề cập như sử dụng học máy và SAGEL với tập địa chỉ tiếng Việt vốn khác rất nhiều so với địa chỉ Ấn Độ về đơn vị hành chính. Ngoài ra, tôi có thử nghiệm một phương pháp kết hợp học máy và SAGEL để có thể khắc phục các nhược điểm của các phương pháp trên với tiếng Việt.







