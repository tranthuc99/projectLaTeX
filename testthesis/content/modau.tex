\chapter*{ĐỀ TÀI TỐT NGHIỆP}

Biểu mẫu của Đề tài/khóa luận tốt nghiệp theo qui định của viện, tuy nhiên cần đảm bảo giáo viên giao đề tài ký và ghi rõ họ và tên.
\newline

Trường hợp có 2 giáo viên hướng dẫn thì sẽ cùng ký tên.\\
%\vspace{14cm}
%\begin{flushright}
%{\large Giáo viên hướng dẫn}\\
% Ký và ghi rõ họ tên
%\end{flushright}

\vspace{14cm}
\begin{tabbing}
\hspace{9cm}\=\kill
 ~ \> {\large Giáo viên hướng dẫn}\\
 \hspace{9.5cm}\=\kill
 ~ \> Ký và ghi rõ họ tên
\end{tabbing} 

 



 
\chapter*{Lời cảm ơn}

Đây là mục tùy chọn, nên viết phần cảm ơn ngắn gọn, tránh dùng các từ sáo rỗng, giới hạn trong khoảng 100-150 từ.


\begin{abstract}
Tóm tắt nội dung của đồ án tốt nghiệp trong khoảng tối đa 300 chữ. Phần tóm tắt cần nêu được các ý: vấn đề cần thực hiện; phương pháp thực hiện; công cụ sử dụng (phần mềm, phần cứng…); kết quả của đồ án có phù hợp với các vấn đề đã đặt ra hay không; tính thực tế của đồ án, định hướng phát triển mở rộng của đồ án (nếu có); các kiến thức và kỹ năng mà sinh viên đã đạt được.

\vspace{12cm}
\begin{tabbing}
\hspace{9cm}\=\kill
 ~ \> {\large Sinh viên thực hiện}\\
 \hspace{9.4cm}\=\kill
 ~ \> Ký và ghi rõ họ tên
\end{tabbing}
\end{abstract}




\tableofcontents % mục lục
\listoftables % danh sách các bảng
\listoffigures % danh sách các hình
